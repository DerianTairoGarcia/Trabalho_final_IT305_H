\documentclass[12pt, 3p]{elsarticle}

%% Import packages
\usepackage[nocfg, stdsubgroups]{nomencl}
%\usepackage{setspace}
\usepackage{amsmath, amsfonts}
\usepackage{multirow}
\usepackage{array}
\usepackage{tikz}
\usepackage{pgfplots}
\usepackage{subfig}
\usepackage{threeparttable}
\usepackage{glossaries}
\usepackage{lipsum}

% \usepackage[ruled, vlined, linesnumbered]{algorithm2e}
% \usepackage{hyperref}

% \usepackage{subcaption}

\usepackage[most]{tcolorbox}
% \usepackage{multicol} % Multiple columns environment
% \renewcommand*\nompreamble{\begin{multicols}{2}}
% \renewcommand*\nompostamble{\end{multicols}}

\tcbuselibrary{breakable}

\usepackage{xcolor}
\usepackage{booktabs}
\usepackage[margin=10pt]{caption}
\usepackage{colortbl}

\definecolor{t1}{RGB}{239, 35, 60} %red
\definecolor{t2}{RGB}{43, 45, 66} %dark grey
\definecolor{t3}{RGB}{141, 153, 174} %light grey

\usetikzlibrary{arrows.meta}
\usetikzlibrary{datavisualization.formats.functions}

%% Defining and loading style of the glossary

\setacronymstyle{long-short}
\loadglsentries{3-acronyms.tex}

\pgfplotsset{compat=1.18}

\begin{document}

\begin{frontmatter}

    \title{Voltage Magnitude Impact of Residential Electric Vehicle Charging: 
    A Case Study in Cusco, Peru}

    \author[1]{Derian Carlos Tairo Garcia}
    \ead{d255905@dac.unicamp.br}

    \author[1]{Erick Alberto Somocurcio Holguin} 
    \ead{e290883@dac.unicamp.br}

    % \author[2]{Juan Camilo López}
    % \ead{j.c.lopezamezquita@utwente.nl}

    % \author[1]{Marcos J. Rider}
    % \ead{mjrider@unicamp.br}

    % \cortext[corr1]{Corresponding author}

    \affiliation[1]{organization={Department of Systems and Energy (DSE), 
        School of Electrical and Computer Engineering (FEEC), 
        State University of Campinas (UNICAMP)},
        city={Campinas},
        state={São Paulo},
        country={Brazil}}
    
    \begin{abstract}
        The integration of \glspl{der}, \glspl{bess}, \gls{pv} systems, 
        and \gls{ev} chargers, introduces new challenges for \gls{ems} in
        microgrids.
        One key challenge is developing an optimized day-ahead EMS specifically 
        tailored for three-phase unbalanced AC microgrids, while also 
        accounting for uncertainties in PV generation and demand. 
        Additionally, the EMS must prepare the microgrid for potential 
        transitions between grid-connected and islanded modes due to 
        unexpected grid outages.
        This study addresses a \gls{moop} approach aimed at minimizing 
        operational costs from the main grid and \gls{ens} for \glspl{ev} 
        in microgrids. 
        We also introduce a fairness model for EV charging to ensure 
        equitable distribution among connected vehicles, considering factors 
        such as the \gls{soc} and availability times at \gls{evcs}.
        The proposed \gls{moop} approach is validated through real-time 
        Hardware-in-the-Loop (HIL) simulations, accounting for multiples 
        contingencies and uncertainties.
        To evaluate the proposed EMS, actual data from the \Gls{campus} 
        at the \gls{unicamp} was utilized. 
        The proposed \gls{minlp} model undergoes a
        transformation into a \gls{milp} model through a 
        series of linearizations. Results confirm the robustness and efficacy 
        of the proposed \gls{ems} in improving the performance and resilience 
        of three-phase unbalanced AC microgrids.
    \end{abstract}
    \begin{keyword}
        Electric vehicles, energy management systems, microgrid, 
        multi-objective approach.
    \end{keyword}
%% End first page  
\end{frontmatter}

%% Start the next pages

\glsresetall

\section{Introduction}\label{sec:intro}


To demonstrate the effectiveness of the proposed solutions, we developed 
six case studies...

After this introduction, the paper is organized as follows:
Section \ref{sec:methodology}, describes the methodology, detailing  
Section \ref{sec:cases_studies}, Section \ref{sec:results} 
present the case studies and results. 
Finally, section \ref{sec:conclusions} concludes the paper, 
highlighting the main contributions.

\section{Methodology}\label{sec:methodology}

This section explains the contingency and uncertainty sets that the \gls{ems} 
mathematical model considered. Additionally, it addressed the fairness criteria 
applied to balance charging among \glspl{ev}, ensuring fair energy distribution. 
Lastly, it thoroughly discussed the solution methodology for 
solving the \gls{moop}, including the optimization approach and computational 
techniques.

\section{Cases studies}\label{sec:cases_studies}

\section{Results}\label{sec:results}

\section{Conclusions}\label{sec:conclusions}

% \appendix

% \section{Repository}\label{appndx:repository}


%% Include the references
% \bibliographystyle{elsarticle-num}
% \bibliography{5-references.bib}

\end{document}